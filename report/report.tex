%%%%%%%%%%%%%%%%%%%%%%%%%%%%%%%%%%%%%%%%%
% University Assignment Title Page 
% LaTeX Template
% Version 1.0 (27/12/12)
%
% This template has been downloaded from:
% http://www.LaTeXTemplates.com
%
% Original author:
% WikiBooks (http://en.wikibooks.org/wiki/LaTeX/Title_Creation)
%
% License:
% CC BY-NC-SA 3.0 (http://creativecommons.org/licenses/by-nc-sa/3.0/)
%
%%%%%%%%%%%%%%%%%%%%%%%%%%%%%%%%%%%%%%%%%
%\title{Title page with logo}
%----------------------------------------------------------------------------------------
%	PACKAGES AND OTHER DOCUMENT CONFIGURATIONS
%----------------------------------------------------------------------------------------

\documentclass[12pt]{article}
\usepackage[english]{babel}
\usepackage[utf8]{inputenc}
\usepackage{amsmath}
\usepackage{color}
\usepackage[explicit]{titlesec}
\usepackage[hyphens,spaces,obeyspaces]{url}
\usepackage{graphicx}
\usepackage{caption}
\usepackage{subcaption}
\usepackage{grffile}
\usepackage{listings}
\usepackage{placeins}
\usepackage[ 
    urldate=long, 
    sorting=none 
]{biblatex} 
\addbibresource{mybib.bib} 

\usepackage{booktabs}
\usepackage{tabularx}

\begin{document}

\begin{titlepage}

\newcommand{\HRule}{\rule{\linewidth}{0.5mm}} % Defines a new command for the horizontal lines, change thickness here

\center % Center everything on the page
 
%----------------------------------------------------------------------------------------
%	HEADING SECTIONS
%----------------------------------------------------------------------------------------

\textsc{\LARGE University of St Andrews}\\[1.5cm] % Name of your university/college
\textsc{\Large Computer Graphics}\\[0.5cm] % Major heading such as course name
\textsc{\large CS4102}\\[0.5cm] % Minor heading such as course title

%----------------------------------------------------------------------------------------
%	TITLE SECTION
%----------------------------------------------------------------------------------------

\HRule \\[0.4cm]
{ \huge \bfseries Ring-Based Distributed System}\\[0.4cm] % Title of your document
\HRule \\[1.5cm]
 
%----------------------------------------------------------------------------------------
%	AUTHOR SECTION
%----------------------------------------------------------------------------------------


\Large \emph{Author:}\\
 \textsc{150008022}\\[1cm] % Your name
 
%----------------------------------------------------------------------------------------
%	DATE SECTION
%----------------------------------------------------------------------------------------

{\large \today}\\[2cm] % Date, change the \today to a set date if you want to be precise

%----------------------------------------------------------------------------------------
%	LOGO SECTION
%---------------------------------------------------------------------------------------

\includegraphics[width = 4cm]{images/standrewslogo.png}
 
%----------------------------------------------------------------------------------------

\vfill % Fill the rest of the page with whitespace

\end{titlepage}

\section*{Goal}

The aim of this practical is to understand the key principles behind various techniques
frequently used for the rendering of 3D objects, and to get hands-on experience with their implementation and manipulation.

\pagenumbering{arabic}
\setcounter{page}{1} 

\section{Running the Program}

The project uses \texttt{maven} for lifecycle and dependency management.
\texttt{LWJGL} was used for this project, which provides a way to use \texttt{OpenGL} from Java. 
OpenGL \textit{may} be a cause of compatibility issues when running, but an early and forward compatible profile was used to minimise the risk of this.

\renewcommand{\arraystretch}{1.5}
\begin{table}[!ht]
\centering
\begin{tabular}{ll}
    \multicolumn{2}{l}{Usage: \texttt{java -jar FaceModelling-jar-with-dependencies.jar}}
\\ 

 -d,--drop           &  Include if this node should trigger a database refresh. \\
\end{tabular}
\caption{Arguments for running application.}
\label{tbl:arguments}
\end{table}

%\begin{figure}[!ht]
%	\centering
%	\includegraphics[width=\linewidth]{images/chang}
%	\caption{Logs from node 8 when it joins the ring and notices its ID is greater than the current coordinator.}
%	\label{fig:chang}
%\end{figure}

\section{Implementation}

% ENABLE_DEPTH_TEST solves same problem as painters algorithm, but uses z-buffers

\printbibliography

\end{document}
